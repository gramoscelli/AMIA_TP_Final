\documentclass{beamer}

\usepackage{subcaption}
\usepackage{graphicx}
\usepackage{caption}
\setbeamertemplate{caption}[numbered]
\usepackage{pgfplots}
\pgfplotsset{compat=1.18}

\usepackage{hyphenat}

\usepackage{listings}  % Para mostrar código

\usepackage{array}
\usepackage{enumitem}
\usepackage{ragged2e}
\usepackage{multicol}
\usepackage{booktabs}
\usepackage{comment}

\usepackage{animate}
\usepackage{colortbl}

\usepackage{fancybox} 
\usepackage{eurosym} 


\usepackage{amsfonts}
\usepackage{amssymb}
\usepackage{amsmath}

% Tick mark
\newcommand{\tick}{\checkmark}
% Cross mark
\newcommand{\cross}{\text{\sffamily X}}

\usepackage[style=numeric,sorting=none]{biblatex}
% Specify the .bib file
\addbibresource{lib.bib}

\usepackage{xcolor}


% Redefine the footnote counter format to use symbols
\renewcommand{\thefootnote}{\fnsymbol{footnote}}



\captionsetup[figure]{font=scriptsize}
\captionsetup[subfigure]{font=tiny}
\captionsetup[table]{font=scriptsize}

% Theme selection
\usetheme{Madrid} % Choose the theme (e.g., Madrid, Berlin, CambridgeUS, etc.)
\usecolortheme{default} % Choose the color theme (e.g., default, albatross, crane, etc.)

\definecolor{pastelgreen}{RGB}{90, 140, 160}
\setbeamercolor{structure}{fg=pastelgreen}


% Definición del estilo para el código
\lstdefinestyle{pythonstyle}{
    backgroundcolor=\color{gray!10},   % Fondo gris claro
    basicstyle=\ttfamily\tiny\color{black},  % Texto más pequeño y fuente tipo máquina
    keywordstyle=\color{blue}\bfseries,  % Palabras clave en azul y negritas
    commentstyle=\color{green}\itshape,  % Comentarios en verde y en itálicas
    stringstyle=\color{orange},   % Cadenas en naranja
    identifierstyle=\color{black},   % Identificadores en negro
    showstringspaces=false,     % No mostrar espacios en las cadenas
    breaklines=true,            % Romper líneas largas
    breakatwhitespace=true,     % Romper líneas solo en espacios
    frame=single,               % Marco alrededor del código
    xleftmargin=10pt,           % Márgenes a la izquierda
    xrightmargin=15pt,          % Márgenes a la derecha
    aboveskip=1em,              % Espacio por encima del código
    belowskip=1em,              % Espacio por debajo del código
    commentstyle=\color{green}, % Comentarios en verde
    morekeywords={self},        % Resaltar 'self' también
    tabsize=4,                  % Tamaño de tabulación
}


% Title page information
\title{TP Final - Análisis Matemático}
\author{GRUPO 2}
\date{\today}

% Presentation content
\begin{document}


% Slide ---------------------------------------------------------------------------------------------------------------------------------
% ---------------------------------------------------------------------------------------------------------------------------------------
%\begin{frame}
 %   \titlepage % Display the title page
%\end{frame}

% Slide ---------------------------------------------------------------------------------------------------------------------------------
%----------------------------------------------------------------------------------------------------------------------------------------
%\begin{frame}
%    \scriptsize
%    \frametitle{Outline}
%    \tableofcontents % Display the table of contents
%\end{frame}

    
% Slide 1
\begin{frame}
  \frametitle{Introducción al problema}

  \begin{itemize}
    \item Comparativa de métodos para mejorar la eficiencia de predicción:
    \begin{itemize}
      \item \texttt{QDA}: Métodos estándar con ciclos \texttt{for}.
      \item \texttt{TensorizedQDA}: Uso de tensores para mayor rapidez.
      \item \texttt{FasterQDA}: Implementación optimizada sin ciclos.
    \end{itemize}
    \item Propósito: Acelerar la predicción eliminando ciclos innecesarios.
  \end{itemize}

\end{frame}

% Slide 2
\begin{frame}
  \frametitle{\texttt{QDA}: Implementación básica}

  \begin{itemize}
    \item Utiliza el método \texttt{\_predict\_one(x)}:
    \begin{itemize}
      \item Recorre cada clase (\texttt{for} interno).
      \item Método iterativo para calcular la verosimilitud.
    \end{itemize}
    \item Limitación: Ineficiente para grandes volúmenes de datos.
    \item Tiempo: Lento debido a múltiples ciclos.
  \end{itemize}

\end{frame}

% Slide 3
\begin{frame}
  \frametitle{\texttt{TensorizedQDA}: Uso de tensores}

  \begin{itemize}
    \item Expande el set de predicción \( X \in \mathbb{R}^{p \times n} \):
    \[
    X_{\text{expanded}} \in \mathbb{R}^{\text{clases} \times p \times n}
    \]
    \item Los promedios se almacenan en:
    \[
    \text{tensor\_means} \in \mathbb{R}^{\text{clases} \times p \times 1}
    \]
    \item Permite operaciones matriciales sin ciclos adicionales.
    \item Mejora significativa en eficiencia.
  \end{itemize}

\end{frame}

% Slide 4
\begin{frame}
  \frametitle{\texttt{FasterQDA}: Optimización total}

  \begin{itemize}
    \item Cálculo del producto interno necesario para la verosimilitud:
    \[
    \text{inner\_prod\_mat} = \text{unbiased\_X\_transposed} \, @ \, \text{self.tensor\_inv\_cov} \, @ \, \text{unbiased\_X}
    \]
    \item Diagonalización para reducir dimensiones.
    \item Uso del determinante de la matriz de covarianza:
    \[
    \frac{1}{2} \log |\Sigma_j^{-1}|
    \]
    \item Todo en una sola pasada sin ciclos.
  \end{itemize}

\end{frame}

% Slide 5
\begin{frame}
  \frametitle{Comparativa de Tiempos}

  \begin{itemize}
    \item Resultados experimentales:
    \begin{itemize}
      \item \texttt{QDA}: 
      \[
      \text{Tiempo: } 0.00242 \, \text{s} \quad (\text{std: } 0.00055)
      \]
      \item \texttt{TensorizedQDA}: 
      \[
      \text{Tiempo: } 0.00062 \, \text{s} \quad (\text{std: } 0.00051)
      \]
      \item \texttt{FasterQDA}: 
      \[
      \text{Tiempo: } 0.00010 \, \text{s} \quad (\text{std: } 0.00030)
      \]
    \end{itemize}
    \item Aceleración significativa al eliminar ciclos.
  \end{itemize}

\end{frame}
\end{document}
